% Run with pdfLaTeX
\documentclass[11pt]{article} %%%

\usepackage[text={6.5in,9in}, top=1in, left=1in]{geometry}
\usepackage{color}
\usepackage{amssymb}
\usepackage{amsmath}
\usepackage{txfonts}
\usepackage{graphicx} %%% use 'pdftex' instead of 'dvips' for PDF output
\frenchspacing\flushbottom

% You can include more LaTeX packages here 

\begin{document}

\title{\bf INFORMS O.R. \& Analytics\\ Student Team Competition 2017:\\
Instructions and Entry~Form}

%% Entry Number will be provided to you by October 30.
\date{\vspace*{-30pt}}

\maketitle

\subsection*{\textcolor{red}{Instructions to Team}}

{\baselineskip14.7pt

\begin{enumerate}

\item Provide your entry following the format in this LaTeX document (the entry form begins on page~3). Maintain the section headings in the form, and replace the explanatory text with your presentation. Please use the same font style and size, margins and line spacing as used in the form.

\item Figures, visualizations, appendices etc. should be embedded in this form, not provided as separate files.

\item Your entry form should not exceed 30 pages, including all figures and appendices.

\item Teams may use any inanimate source of data or materials: computers, software, references, web sites, books, etc. All sources must be credited. Failure to credit a source will result in disqualification. Follow instructions for reference citation provided at the end of this form. 

\item Do not include team member names, the university name, or other identifying information. Your entry number is the identification for your entry. Your entry number will be provided to you. If you do not receive your number by October 30, please email \underline {orastc@mail.informs.org}.

\item Entries are due no later than midnight US Eastern Time, January 30, 2017. No further modifications, enhancements, additions or improvements may be made to the team's entry after January 30. 

\item Entries must be submitted electronically; you will receive instructions on where to upload your entry. 

\item Failing to do any of the above may result in disqualification.

\end{enumerate}


\subsection*{\textcolor{red}{Judging Criteria}}

Entries will be judged on use of the full analytics process---incorporating 
not only the technical analysis and solution, but also understanding of the 
business problem, team organization, and the clarity and effectiveness of 
communication. The written report your team provides in this entry form is 
the foundation of your entry. Think of this report as a presentation to 
Syngenta's upper management on the critical elements in the analytics 
decision process. 

\begin{itemize}

\item The report is cleanly written (few typos; few distracting grammatical hiccups; appendices, tables and figures are well-labeled and easy to read).

\item The writing is coherent, fluid, and has a clear structure.

\item The Executive Summary is appropriately written for the target audience (Syngenta executives). 

\item The team makeup and process are clearly described so that the reader can understand the division of labor and workflow of the project.

\item The problem is clearly defined. Relevant stakeholders are identified, and their respective objectives and tradeoffs are addressed and weighed.

\item Data used in the model have been collected, cleaned and analyzed correctly.

\item The methodology used for solving the problem is reasonable.

\item The model has been clearly formulated, with decision variables and parameters defined. The model formulation is correct and applicable to the problem.

\item Simplifying assumptions made in the model are clearly stated, reasonable, and assessed.

\item The report proposes a solution to the problem that makes sense and adds value to Syngenta. The report assesses how well the solution meets the criteria of the stakeholders. The analysis assesses how sensitive the solution is to assumptions and data. 

\end{itemize}

}

\vfill\eject

\begin{centering}
\LARGE\bf
INFORMS O.R. \& Analytics Student Team Competition 2017\\
ENTRY FORM\\[10pt]
%% Entry Number will be provided to you by October 30.
{\Large\bf Entry Number:}\\[-5pt]
{\normalsize\it Entry Number will be provided to you by October 30.}\\
\end{centering}

\begingroup
\baselineskip 17 pt plus 2pt minus 2pt

\subsection*{Executive Summary (not to exceed 2 pages)}

In this section, provide an executive summary for Syngenta executives. Your 
summary should briefly address your understanding of the business problem 
and how you approached solving it, covering not only the technical decisions 
and analysis but also the process your team used. The summary should also 
briefly describe your recommendations to Syngenta.

\subsection*{Team Makeup \& Process}

Without providing the names of individuals, describe the makeup of your team 
and the process your team used in working on the problem. This may include 
team members' background and experience, particularly as they may relate to 
the role each member played in the project. In addition, describe the 
process your team used, including elements such as how the work was 
allocated, how the team demonstrated the value of team members' ideas, how 
work from individuals was synthesized into the final analysis. You can also 
address any challenges and learnings from the team experience. 

\subsection*{Framing the Problem}

Describe your understanding of the business problem presented by Syngenta 
through the written problem statement, the webinar with company executives, 
and their answers to questions your team posed. Then explain how your team 
reformulated the business problem into an analytics problem amenable to an 
analytics solution. This may include delineating constraints, as well as 
defining a set of assumptions and key metrics of success. 

\subsection*{Data}

Data for the problem was provided by Syngenta. If you used other data or 
sources, please define them here. Describe any work your team may have done 
with the data itself, such as rescaling, cleaning, identifying 
relationships, etc. If your analysis of the data helped to refine your 
understanding of the business and/or analytics problem, describe that here.

\subsection*{Methodology Approach \& Model Building}

In this section, describe the decision making process for selecting an 
analytics methodology(ies). What other methodologies did your team consider 
and what were the reasons for the final selection? You may want to include 
discussion and considerations -- such as the assumptions that were made, the 
scope and early considerations---to provide a useful framing for your 
selection. Then describe and document the chosen methodology and model in 
sufficient detail and clarity that it can be understood and evaluated. Your 
selection of software should also be addressed here. Given the 
multi-disciplinary aspect of the problem, background information may be 
useful to include or reference.

\subsection*{Quantitative Results}

In accordance with your methodology and model, present quantitative results 
here. You may supplement this section with charts, diagrams and/or other 
visualizations, which must be included in this document. 

\subsection*{References}

Please follow guidelines in the \textit{Chicago Manual of Style}, 16$^{th}$ Edition. Here are examples: 

\begin{enumerate}

\item Journal article: Flynn J, Gartska SK (1990) A dynamic inventory model with periodic auditing. \textit{Oper. Res.} 38(6):1089--1103. 

\item Book: Makridakis S, Wheelwright SC, McGee VE (1983) \textit{Forecasting: Methods and Applications}, 2nd ed. (John Wiley {\&} Sons, New York). 

\item Edited Book: Martello S, Toth P (1979) The 0-1 knapsack problem. Christofides N, Mingozzi A, Sandi C, eds. \textit{Combinatorial Optimization} (John Wiley {\&} Sons, New York), 237--279.

\item Online reference, fictional example: American Mathematical Institute (2005) Better predictors of geospatial variability. Retrieved June 14, 2005, \underline {www.mathematicsinstitute}.

\end{enumerate}

\endgroup

\end{document}

